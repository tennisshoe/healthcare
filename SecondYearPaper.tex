\documentclass[12pt]{article}

\usepackage{amsmath} 
\usepackage{verbatim}
\usepackage{amsfonts}
\usepackage{indentfirst}
\usepackage[shortlabels]{enumitem}
\usepackage{graphicx}
\usepackage{epstopdf}
\usepackage{float}
\usepackage[T1]{fontenc}
\usepackage{lmodern}
\usepackage{subcaption}

\addtolength{\textwidth}{2cm}
\addtolength{\hoffset}{-1cm}
\addtolength{\textheight}{2cm}
\addtolength{\voffset}{-1cm}

\pagestyle{myheadings}
\markright{\hfill TIES Second Year Paper \hfill}
\title{Healthcare as an Entrepreneurial Subsidy}
\author{Ankur Chavda}

\begin{document}
\maketitle

\begin{abstract}
We study the effect of the introduction of public health insurance on entrepreneurial rates. In 2006, Massachusetts legislated a health care market which facilitated the purchase of health insurance by individuals through an agency of the state government. We test whether this change in availability of health insurance affected entrepreneurship. We find a statistically significant increase in small establishments for industries that require low levels of startup capital.
\end{abstract}

\section{notes}

freelancers not interesting per se but cases where they grow into growth startups do exist (GetHarvest). 

check to make sure model (4) is referred to as being MA only

check law journals for massachusetts health care papers

\section{Introduction}

Anecdotal evidence suggests health care is an important factor in the decision of whether to leave a firm and become self-employed. To quote one entrepreneur: 
\begin{quote}
Obamacare affected me in another critical way as well. Its assurance of a stable insurance market that does not screen out someone with a pre-existing condition made me far more comfortable starting my own business. It gave me a baseline of security that simply didn't exist before. It helped make entrepreneurialism possible. \cite{sullivan}
\end{quote}
If this is the case, changes in access to health care might constitute a shock to the potential entrepreneur's decision process. Entrepreneurs who have the ideas and resources to start companies may be nudged into self-employment by the increased availability of individual health care plans outside the usual employer provided system in the United States. 

We suggest using the health insurance reforms passed in 2006 by the Massachusetts legislator as such a shock on self-employment. Although this limits us to an empirical setting of just one state, the similarities between the Massachusetts reforms and the country-wide Affordable Care Act should make any results more broadly applicable. In the economic literature Miller \cite{miller} was the first to use the shock, specifically on studying health outcomes. To our knowledge we are the first to use the shock to measure self-employment outcomes. 

Understanding the relationship between such a policy change and self-employment would be helpful in a number of contexts. First, the relationship can inform whether self-employment is facilitated by short term incentives that make starting a company easier. Second, proper matching between workers and employers has been a growing concern of business in recent years; for an example consider Amazon's willingness to pay workers \$5000 to quit their jobs \cite{bezos}. Enabling firm employees to make better employment choices can increase overall worker productivity by reducing job lock.  Last, economists can incorporate how access to health care promotes entrepreneurship in any partial equilibrium analysis used to set optimal health policy. 

\section{Research Contribution}

There are four main research literatures we build upon: job lock, taxation, entrepreneurship and firm entry.

Within the job lock literature there has been a long history of testing how health care changes incentives to leave employment. Gruber and Madrain find that enabling employees to keep their current policy reduces job lock \cite{gruberMadrian}. More recently, 
Garthwait, Gross and Notowidigo show how the loss of healthcare to a group in Tennesee increased the area's labor supply \cite{garthwaite}. 

There is also a growing body of research that specifically tests how health care affects self-employment. Heim and Lurie analyze tax filing data and find that the health act in Massachusetts caused a drop in workers that spend the majority of time self-employed \cite{heimLurie} while Tuzemen and Becker suggest the opposite \cite{tuzemen}. Bailey finds no like between the Affordable Care Act's Dependant Coverage mandate and entrepreneurial rates \cite{bailey}. We complement this set of research in two ways. First, we consider whether drop in majority time self-employed corresponds with an increase in firm creation. This would be the case if for example the health insurance reform encouraged existing self-employed workers to incorporate firms. Second, we investigate the marginal effect on different groups of entrepreneurs. We provide evidence that the effect is heterogeneous which may explain why different authors have seen different results in the past.

Labor models of taxation for self-employment typically use some form of a Roy model \cite{roy} to consider individuals with heterogeneous tastes or skill for two sectors: entrepreneurship and labor. Scheuer \cite{scheuer} for example explores optimum tax policy with individuals of two dimensional type. The taxation literature has also explored how workers might value benefits that are funded by taxes, see Gruber \cite{gruber1997}. We contribute by providing a novel setting to test the relationship between expected taxation and self-employment. 

Entrepreneurial scholars view self-employment as a core mechanism behind Schumpeter's evolutionary process of capitalism \cite{schumpeter}. Under the assumption that the benefits of capitalistic markets are increasing in number of entrepreneurs, a line of research has focused on ways to increase the supply of entrepreneurs. One line of research has investigated how larger firms create entrepreneurs and focuses on the characteristics of how these new ventures are formed out of other firms\cite{burton} \cite{gompers}. Another line of research considers how for example access to computers impacts the likelihood of an individual becoming an entrepreneur \cite{fairlie}. We contribute more to this second line of research. Farilie, Karan and Zinman found a connection between training and entrepreneurship conditional on access to capital \cite{fairlieKarlanZinman}. We take advantage of this insight by investigating how health care and entrepreneurship might be linked by access to capital. 

The industrial organization literature on firm entry tends to focus on how other firms affect a new firm's success in the market. Classical models like the Ericson and Pakes model considers how competition impacts the decision of a firm to further invest and stay in a market \cite{ericsonPakes}. Jovanovic's model  focuses on the path a firm takes after entry as it learns about its cost structure \cite{jovanovic}. Although Hopenhayn moves a little farther backwards in time and considers a firms entry decision as a function of expected profit, the model views the resource constrains on an entrepreneur as exogenous \cite{hopenhayn}. Klepper can be thought of as endogenizing these resource through a single measure of innovative capability \cite{klepper}. The model we develop attempts to break out resources separately and describe how various constraints interact with each other for an individual entrepreneur. Our model is an intermediate stage between the idea draw and the realization of an investment outcome; it can be viewed as nested within a model like Klepper's model. 

\section{Empirical Setting}

The research would explore whether the Massachusetts health care reform of 2006 changed the rate of new firm creation. In late 2006, the state of Massachusetts passed a law entitled ``An Act Providing Access to Affordable, Quality, Accountable Health Care''. The act was motivated by a number of concerns, including:

\begin{enumerate}
\item Reducing the number of uninsured residents, which had been steadily increasing in the period from 2000 to 2004 \cite{bisweek}
\item Addressing the public cost of providing healthcare to uninsured residents \cite{npr}
\item Restructuring a federal waiver that allocated funds to Massachusetts for uncompensated care \cite{heritage}
\item The highest individual (non-group) health insurance costs in the nation \cite{gruber}
\end{enumerate}

The act resulted in a number of changes to the health care industry between 2006 and 2007. Provisions included a mandate that required all adult residents to purchase health insurance conditional on affordability guidelines, health insurance subsidies to lower income families, a state run marketplace for individual insurance, a tax on larger employers that declined to provide health insurance to employees and the expansion of the state's Medicaid program for children. Much of the act was phased in over time as seen in Figure 1.

\begin{figure}[H]
	\centering
	\includegraphics[scale=0.75]{resources/timeline}
	\caption{Timeline of plans available for individual purchase and mandate requirements}
\end{figure}

A net effect of the act was a drop in uninsured Massachusetts residents. The Massachusetts Health Insurance Survey is conducted in the first half of each year and asks about the current insurance status of Massachusetts residents. It suggests that the act resulted in an approximate 60\% reduction in uninsured adult residents. Figure 2 illustrates the effect by contrasting the periods before and after the law was fully implemented. 

\begin{figure}[H]
	\centering
	\includegraphics[scale=0.6]{resources/uninsured}
	\caption{Percent of Massachusetts residents uninsured between the ages of 19 to 64}
\end{figure}

There is also a tangible cost to entrepreneurs when universal health care is provided: increased taxes.  While universal health care may have near term benefits for entrepreneurs, it also has long term costs that reduce payoffs. 

From an entrepreneurial standpoint, we view the act as having improved the health insurance marketplace for self-employed residents. We are agnostic to the mechanism. One possibility is the lower cost of insurance; Gruber \cite{gruber} suggests that controlling for plans with lower benefits, the cost of individual insurance plans dropped by 20\% as a results of the reforms. Other potential mechanisms include the ability to sign up for plans with pre-existing conditions and greater confidence in keeping insurance conditional on becoming sick. We consider the act as an exogenous shock on entrepreneurship outside of the healthcare industry since the passage of the act was not driven by entrepreneurial activity in Massachusetts. We use the period starting in 2008 as the treatment period for the shock since the law's main effect of lowering the uninsured rate was observed in 2008. 

\section{Theoretical Motivation}

We will use this empirical context to test four propositions. First, building on Gruber and Madrain \cite{gm2002}, consider a compensating differential model of a worker that currently has health care through their job but expects to be unable to attain health care upon self-employment. Utility is a function of both earnings and a binary indicator of health insurance coverage. Workers stay at the firm as long as employed wages $w$ and the return to self-employment $r$ as such that
$$U(w,1) \ge U(r,0)$$

If there is a shock that creates a market for non-group health insurance at cost $c$, the worker will stay as long as
$$U(w,1) \ge \max\{U(r,0),U(r-c,1)\}$$

This should shift some mass of workers to become self-employed as long as some workers value health insurance more than the cost on the non-group health market and for those workers

$$U(w,1) < U(r-c,1)$$

So this model would predict that the health-care shock increases self-employment. 

\textbf{Proposition 1:} 
Improving access to health-care should increase the self-employed workforce. 

However, the reform in Massachusetts also involved changes to the taxes, including the use of general revenue funds to pay for subsidies and penalties for firms that did not provide health care to workers. In 2013, 30\% of the Massachusetts budget was spent on the Mass Health \cite{masshealth} subsidy program. This may have affected the expected  return $r$ to self-employment conditional on creating a successful firm. For example, entrepreneurs could believe the health care subsidies would eventually be paid for by a progressive tax on earnings.

This could be viewed as a mitigating the effect of the shock on self-employment. Now workers will leave if taxes $t$ are such that

$$U(w,1) < U(r(1-t)-c,1)$$

These expectations should affect non-profit firms differently than for-profit firms and so we should see a heterogenous effect of the shock across these two types of firms. 

\textbf{Proposition 2:} 
The health care reform package should cause a relatively larger increase in non-profit firms than for-profit firms. 

Our next proposition is considers industry based heterogeneity. Suppose a potential entrepreneur is considering to start a new venture. Conditional on having an idea, factors such the perceived opportunity for the idea \cite{sorensenSorensen} and ability to attract complementary talent \cite{stuartSorensen} can impact the decision to execute the venture. We can model the probability of starting a firm as the probability of getting an idea that fits the constraints imposed by the entrepreneur. 

Suppose two such constraints are the capital required to start the firm and access to healthcare  outside of employment. An idea can vary on both dimensions. Some ideas can of course require more capital to execute than others. Additionally, some ideas could be worked on without leaving existing employment. For example, survey data suggests half of all software developers create applications outside of their formal employment \cite{evans}. Also, the constraints are likely to vary by individual. Individuals vary in their social networks which can influence access to capital \cite{uzzi}. We can also imagine health care constraints varying by individual. A potential entrepreneur may be relatively less constrained by health care if her spouse's employer provided health insurance for their family. 

\textbf{Proposition 3:} 
Improving access to healthcare should increase the probability of starting a low capital firm relative to a high capital firm. 

\textbf{Proof:}
See Appendix A for a formal statement of the proposition, the required assumptions and its proof. 

The intuition behind Proposition 1 can be conveyed in the single entrepreneur case using a visualization of the idea space. Suppose an entrepreneur has a level $\alpha_i$ of healthcare and level $\beta_i$ of capital to start with. The entrepreneur can then execute ideas that require at most those levels of resources. The shaded region in Figure 6 represents the space of ideas executable by the entrepreneur. If we think of the healthcare reform as shifting level $\alpha_i$ to $\alpha_i'$, the striped region's set of ideas also become feasible. Note that only ideas with resource requirements lower than $\beta_i$ are affected by the shift in resource $\alpha$. As long as there is not a larger mass of potential ideas that require high capital than low capital, when we sum across all individual entrepreneurs we should see more executed ideas with low capital than high capital. 

\begin{figure}[H]
	\centering
	\includegraphics[scale=0.5]{resources/Prop1}
	\caption{Subset of idea space executable by agent $i$ with resources $(\alpha_i,\beta_i)$ versus$(\alpha_i', \beta_i)$.}
\end{figure}

Figure 7 provides some evidence that ideas requiring low capital are more common than ideas requiring high capital. It plots all active Kickstarter projects by funding goal. We see the density of projects generally decreasing in capital. Violations of monotonocity can reasonably be accounted for by rounding behavior; for ideas requiring capital amounts near \$50,000, entrepreneurs round to \$50,000 when listing their project goals. 

\begin{figure}[H]
	\centering
	\includegraphics[scale=0.75]{resources/kickstarter}
	\caption{Funding goal density plot of active Kickstarter project active on April 2nd, 2015 }
\end{figure}

Our second proposition describes the relative short and long term affect of a health care shock. 

\textbf{Proposition 4:} The relative increase in low versus high capital firms should attenuate over time. 

\textbf{Proof:} See Appendix A for a formal statement and proof

This follows from entrepreneurs having any memory of past ideas. Immediately after the shock, ideas in Figure 7's striped region that occurred to the entrepreneur in the current time period as well as prior periods become feasible. In future periods, feasible ideas are only drawn from the new ideas generated by the entrepreneur that period.

\section{Data Sources}

Our primary source of data on entrepreneurship comes from Census Country Business Patterns (CBP) data. The CBP is an annual series based on the Census Business Register, a dataset of US firms that is continuously updated using multiple methods including surveys and tax reporting. The CBP provides country level information on for example the industry, sales and employee count of establishments. It excludes businesses with no paid employees. We use establishments of 1 to 4 employees as a proxy for entrepreneurial firms. Table 1 provides summary means for small establishments in Massachusetts and other New England states based on the CBP and Census population data from 2000 to 2012. To reduce potential endogeneity we remove firms in the health care industry from our sample since our shock may have increased health related opportunities for entrepreneurs. 


\begin{flushleft}
\begin{tabular}{|p{6cm} | r | r| r| r|}
\hline
& \multicolumn{2}{ |c| }{Massachusetts} & \multicolumn{2}{ |c| }{Other New England}\\
\hline
& \multicolumn{1}{ |c| }{Prior to} & \multicolumn{1}{ |c| }{Post}  & \multicolumn{1}{ |c| }{Prior to} & \multicolumn{1}{ |c| }{Post} 	\\
& Treatment & Treatment  & Treatment & Treatment \\
\hline
\begin{tabular}{lrr} \hline \hline
& Massachusetts  & Other New England  \\  \hline
Population &   6,431,559 &    7,847,646 \\
Percent Urban &        91.09\% &        60.91\% \\
Percent Uninsured &        13.50\% &        13.74\% \\
Percent Aged 20 to 24 &         6.40\% &         5.94\% \\
Percent of Households with Children &        30.60\% &        31.68\% \\
\hline \textbf{Self-Employment} & & \\
Number &      210,684 &      311,317\\
Percent of Population &       3.28\% &       4.01\% \\
Yearly Change in Percent of Population &         -0.01\% &         -0.02\% \\
\hline \textbf{Non-employers} & & \\
Number &      423,846 &      572,954\\
Percent of Population &       6.39 \% &       7.15 \% \\
Yearly Change in Percent of Population &          0.06 \% &          0.07 \% \\
 \hline \hline \end{tabular}

\hline
\end{tabular}
\end{flushleft}

The rest of New England is approximately similar to Massachusetts in size and gross establishment characteristics. The summary means suggest that there was a reduction in net entrepreneurial activity across New England after the time of the health care shock. Before 2008, about 600 net new entrepreneurial firms were recorded in Massachusetts per year. From 2008 to 2012 there was on average about 1200 fewer entrepreneurial firms per year. Figure 3 provides additional detail about entrepreneurial firms by industries, illustrating that the distribution of entrepreneurship across industries is similar between Massachusetts and the rest of New England.  

\begin{figure}[H]
	\centering
	\includegraphics[scale=.75]{resources/naics_summary}
	\caption{Average yearly change in number of small establishments per 1M residents}
\end{figure}

Figure 4 shows by county the change in number of small establishments per 1M residents both before and after the treatment. We see a similar pattern across Massachusetts counties and the rest of New England. 

\begin{figure}[H]
	\centering
	\includegraphics[scale=.75]{resources/county_summary}
	\caption{Average yearly change in number of small establishments per 1M residents}
\end{figure}


Additionally we use data from the Census Survey of Business Owners (SBO) to understand characteristics of entrepreneurial firms. Table 3 provides for example the number of firm owners indicating no startup capital was required to start the business.  It includes all non-farm businesses with revenue greater than \$1000 per year that file with the IRS. Unlike the CBP, the SBO includes firms with no employees.

\begin{flushleft}
\scalebox{.9}{
\begin{tabular}{|c|l|r|}
\hline
NAICS & Industry & No funding\\
\hline
61 & Educational Services &	19.2\%\\
71 & Arts, Entertainment, and Recreation  &	17.9\%\\
22 & Utilities	 & 16.8\%\\
55 & Management of Companies and Enterprises &	15.4\%\\
52 & Finance and Insurance &	14.5\%\\
54 & Professional, Scientific and Technical Services &	12.1\%\\
21 & Mining, Quarrying, and Oil and Gas Extraction & 	11.9\%\\
23 & Construction &	11.8\%\\
51 & Information &	11.8\%\\
62 & Health Care and Social Assistance &	11.8\%\\
53 & Real Estate and Rental and Leasing &	11.6\%\\
56 & Administrative, Support, Waste Management, Remediation Services &	11.0\%\\
11 & Agriculture, Forestry, Fishing and Hunting &	10.2\%\\
42 & Wholesale Trade &	10.1\%\\
48-49 & Transportation and Warehousing &	9.3\%\\
31-33 & Manufacturing &	9.1\%\\
81 & Other Services, except Public Administration &	8.7\%\\
44-45 & Retail Trade &	6.9\%\\
72 & Accommodation and Food Services &	5.3\%\\
\hline
\end{tabular}
}
\end{flushleft}

Finally, we use the 2012 Economic Census Industry Series for information about the concentration of non-profits per industry. The data classifies firms by whether they are exempt from federal income tax in some service industries. We exclude the subset of industries relating to health care services.

\section{Empirical Results}
In order to measure the effect of the health care reform on entrepreneurship, ideally we would want to measure change in probability an individual becomes an entrepreneur. Since we do not observe new firm creation, here we use the yearly number of net new small establishments per million people as a proxy for the probability of entrepreneurship. In some of our specifications, we measure this by county ($NewNew_{ct}$) while in others we measure rates per 2 or 4 digit NAICS industry by county ($NewNew_{cit}$). This identification strategy mixes both firm starts and firm failures, so any results speak to the impact of the health care reform both on the ability for entrepreneurs to start new businesses as well as keep existing firms afloat. Table 1 provides summary statistics for the variables of interest across the treatment and control groups used in our empirics. 
\begin{flushleft}
\begin{table}[H]
	\centering
	\begin{tabular}{|p{6cm} | r | r| r| r|r|}
	\hline
	\input{resources/long_summary.tex}
	\hline
	\end{tabular}
	\caption{Estimation Summary Statistics}
\end{table}
\end{flushleft}

\subsection{General Impact}

To test the first proposition, we apply a diff-in-diff approach with counties in Massachusetts as the control group and counties in the rest of New England as control.
\begin{align}
NetNew_{ct} = \alpha_c + \delta_t + \beta \, \mathbf{1}\{\text{county in MA}\} \cdot \mathbf{1}\{\text{year > 2007}\} + \epsilon_{ct}
\end{align}

Here $NetNew_{ct}$ represents the number of net new small establishments created per 1,000,000 residents in year $t$ and county $c$. The model includes fixed effects for year ($\delta$) and county ($\alpha$). We calculate robust standard errors clustered on county and weigh each county by its population at the start of the treatment. Figure 5a shows the difference between the control and treatment groups over the period of interest conditional on the fixed effects. Model (1) is estimated in Table 2\footnote{Robust standard errors clustered on county used under a fixed effect model. Panel series observations weighted by 2008 population. Census county business used to determine number of establishments of 1 to 4 employees by county from 2000 to 2008 for 2 digit NAICS other than 95, 99 and 62. Establishments marked as state wide dropped due to weighting ambiguity. New England considered to be Connecticut, Maine, New Hampshire, Rhode Island, Vermont and Massachusetts. } . 

\begin{figure}[H]
	\centering
	\begin{subfigure}[b]{0.495\textwidth}
	    \includegraphics[width=\textwidth]{resources/state_diff}
	    \caption{Model (1) weighted by county population and clustered by county}
	\end{subfigure}
	\begin{subfigure}[b]{0.495\textwidth}
		  \includegraphics[width=\textwidth]{resources/state_diff_nan}
		  \caption{Model (2) clustered by county and naics}
	\end{subfigure}
	\caption{Fixed effects point estimates of difference between control and treatment groups with 95\% confidence interval.}
\end{figure}

\begin{center}
	\begin{table}[H]		
		\centering
		\input{resources/prop1.tex}	
		\caption{Diff-in-diff estimator of health reform} 
	\end{table}
\end{center}


We are unable to reject the null hypothesis that the 2008 health insurance shock had no effect on overall entrepreneurship using a general difference-in-difference approach.

\subsection{Non-profit Heterogeneity}

For the second proposition, our treatment group consists of 4 digit NAICS code industries where over half of all establishments were exempt from federal taxes, excluding establishments in the health care industry. This specification adds a fixed effect for industry $\gamma_i$ and restricts our observations to counties in Massachusetts. Since our observations are of industries within each county, we no longer weight our results by county. We interpret our coefficient of interest as the number of new establishments per treated industry created per million residents a result of the health care shock. 
\begin{align}
NewNew_{cit} = \alpha_c + \gamma_i + \delta_t + \beta \, \mathbf{1}\{\text{non-profit industry}\} \cdot \mathbf{1}\{\text{year > 2007}\} + \epsilon_{ct}
\end{align}

Here $NewNew_{cit}$ represents the number of net new small establishments created per 1,000,000 residents in year $t$, county $c$ and industry $i$. The model includes fixed effects for year ($\delta$), county ($\alpha$) and industry ($\gamma$). 

Model (2) leaves open the possibility of another shock happening at the same time that also differentially affected non-profit industries. The results would be invalid if for example the 2008 recession was relatively milder for non-profit firms than for for-profit firms. To control for this, we extend our approach to a triple difference model which should account for shocks that both affected Massachusetts and the rest of New England. 
\begin{align}
NewNew_{cit} = & \; \gamma_i \cdot \alpha_c + \gamma_i \cdot \delta_t +  \alpha_c \cdot \delta_t \nonumber   \\
& + \beta \, \mathbf{1}\{\text{non-profit industry}\} \cdot \mathbf{1}\{\text{year > 2007}\}  \cdot \mathbf{1}\{\text{county in  MA}\} \nonumber  \\
& + \epsilon_{ict}
\end{align}

Figure 5b shows the difference between the control and treatment groups over the period of interest conditional on the fixed effects for model (2). Our results for both specifications are presented in Table 3\footnote{Robust standard errors clustered on county and 4 digit NAICS industry used. US wide 2012 Economic Census industry data used determine 4 digit NAICS codes with over 50\% of federal tax exempt establishments. Health care related establishments and state-wide establishments dropped.}. Neither model suggests there was a significant difference between the entrepreneurial rates of non-profit and for-profit firms as a result of the shock. We interpret this as evidence that the long term tax implications of entrepreneurial subsidies are not attenuating self-employment rates. 
\begin{center}
	\begin{table}[H]
		\centering
		{
\def\sym#1{\ifmmode^{#1}\else\(^{#1}\)\fi}
\begin{tabular}{l*{4}{c}}
\hline\hline
          &\multicolumn{1}{c}{(1)}&\multicolumn{1}{c}{(2)}&\multicolumn{1}{c}{(3)}&\multicolumn{1}{c}{(4)}\\
\hline
\% Non-profit $\times$ Post 2007&   42.565***&            &            &            \\
          &  (9.203)   &            &            &            \\
MA $\times$ \% Non-profit $\times$ Post 2007&            &   52.015***&   20.728   &   52.550***\\
          &            & (14.387)   & (18.712)   & (12.941)   \\
County $\times$ Year FE &       No   &      Yes   &      Yes   &      Yes   \\
Industry $\times$ Year FE &       No   &      Yes   &      Yes   &      Yes   \\
County $\times$ Industry FE &       No   &      Yes   &      Yes   &      Yes   \\
Year FE   &      Yes   &       No   &       No   &       No   \\
County FE &      Yes   &       No   &       No   &       No   \\
State FE  &      Yes   &       No   &       No   &       No   \\
Industry FE &      Yes   &       No   &       No   &       No   \\
\hline
\(N\)     &     3288   &    10560   &     4236   &     6384   \\
\(R^{2}\) &    0.019   &    0.143   &    0.121   &    0.131   \\
Control   &Within MA   &New England   &Border Counties   &    Synth   \\
\hline\hline
\multicolumn{5}{l}{\footnotesize Standard errors in parentheses}\\
\multicolumn{5}{l}{\footnotesize * p<0.10, ** p<0.05, *** p<0.01}\\
\multicolumn{5}{l}{\footnotesize Column (1) is a diff-in-diff model of non-employers per million people in Massachusetts from  }\\ 
\multicolumn{5}{l}{\footnotesize \space 2000 to 2012 with treatment after 2007 and treatment intensity equal to percentage of non-profit}\\ 
\multicolumn{5}{l}{\footnotesize \space firms within industry.}\\ 
\multicolumn{5}{l}{\footnotesize Non-profit industries defined by 4 digit NAICS codes where over half of firms were tax-exempt }\\
\multicolumn{5}{l}{\footnotesize  \space according to 2007 Economic Census data.}\\
\multicolumn{5}{l}{\footnotesize Maine, Connecticut, Vermont, Rhode Island and New Hampshire used as controls in column (2) }\\
\multicolumn{5}{l}{\footnotesize \space under triple diff model. }\\
\multicolumn{5}{l}{\footnotesize Column (3) restricts the dataset to counties that border other states across Massachusetts, }\\
\multicolumn{5}{l}{\footnotesize \space Connecticut, Vermont, Rhode Island, New Hampshire and New York. }\\
\multicolumn{5}{l}{\footnotesize Column (4) uses a synthetic control model that matches each Massachuetts industry by county }\\
\multicolumn{5}{l}{\footnotesize \space pre-trend against US counties with similar income, age, urban and insurance rate characteristics. }\\
\end{tabular}
}

		\caption{Impact of health reform on non-profit entrepreneurship}
	\end{table}
\end{center}

\subsection{Capital Requirement Heterogeneity}

For the third proposition, we can group industries with a high percentage of "no funding" firms together as a treatment group. 
\begin{align}
NewNew_{cit} =  \alpha_c + \gamma_i+ \delta_t + \beta \, \mathbf{1}\{\text{low capital industry}\} \cdot \mathbf{1}\{\text{year >= 2008}\} + \epsilon_{ict}
\end{align}

We need to pick what set of industries constitute low capital from our data set of 18 industries. Figure 8 shows the treatment effect by quantile of the treatment group. For example, the 11th quantile estimate considers the bottom 2 industries by capital requirements as the treatment group. For the rest of our analysis, we use these bottom 2 industries as the treatment group. The treatment effect is emasculated in at higher treatment cutoffs which is consistent with our theory under the assumption that idea density diminishes at higher levels of required capital. This suggests the health care act primarily affected entrepreneurs with low access to capital. 

\begin{figure}[H]
	\centering
	\includegraphics[scale=.75]{resources/quantiles.png}
	\caption{Treatment Effect By Quantile of Industry's Capital Requirements}
\end{figure}


\begin{comment}

\begin{figure}[H]
	\centering
	\begin{subfigure}[b]{0.495\textwidth}
		\includegraphics[width=\textwidth]{resources/quantiles.png}
		\caption{Model (4)}
	\end{subfigure}
	\begin{subfigure}[b]{0.495\textwidth}
		\includegraphics[width=\textwidth]{resources/quantiles_ddd.png}
		\caption{Model (5)}
	\end{subfigure}
	\caption{Treatment Effect By Quantile of Industry's Capital Requirements}
\end{figure}

\end{comment}

Again, model (4) leaves open the possibility of another shock happening at the same time that also differentially affected low capital industries. We extend our approach to a triple difference model (5). 
\begin{align}
NewNew_{cit} = & \; \gamma_i \cdot \alpha_c + \gamma_i \cdot \delta_t +  \alpha_c \cdot \delta_t \nonumber   \\
& + \beta \, \mathbf{1}\{\text{low capital industry}\} \cdot \mathbf{1}\{\text{year >= 2008}\}  \cdot \mathbf{1}\{\text{county in MA}\} \nonumber  \\
& + \epsilon_{cit}
\end{align}

Figure 8b shows model (5) is consistent with our results from model (4). Figure 5b plots the different between control and treatment when we consider the three industries with the lowest capital requirements as the treated group under model (4).  

The fourth proposition would modify the diff-in-diff in that we would test to see if the initial treatment period was stronger than the long term treatment, with a corresponding triple differences estimator.
\begin{align}
NewNew_{cit} & =  \alpha_c + \gamma_i+ \delta_t  \nonumber + \beta_{2008} \, \mathbf{1}\{\text{low capital industry}\} \cdot \mathbf{1}\{\text{year == 2008}\} \nonumber \\
& + \beta_{>2008} \, \mathbf{1}\{\text{low capital industry}\} \cdot \mathbf{1}\{\text{year > 2008}\} + \epsilon_{ict} \\
NewNew_{cit} & =  \; \gamma_i \cdot \alpha_c + \gamma_i \cdot \delta_t +  \alpha_c \cdot \delta_t \nonumber   \\
& + \beta_{2008} \, \mathbf{1}\{\text{low capital industry}\} \cdot \mathbf{1}\{\text{year == 2008}\}  \cdot \mathbf{1}\{\text{county in MA}\} \nonumber  \\
& + \beta_{>2008} \, \mathbf{1}\{\text{low capital industry}\} \cdot \mathbf{1}\{\text{year > 2008}\}  \cdot \mathbf{1}\{\text{county in MA}\} \nonumber  \\
& + \epsilon_{cit}
\end{align}

Table 4\footnote{Robust standard errors clustered on county and 2 digit NAICS industry used. US wide 2007 Survey of Small Business Owners used to determine percentage of firms within 2 digit NAICS industries that required no startup capital.} summarizes our results. All our model specifications show an increase in entrepreneurial activity for low capital industries. On average, model (5) suggests about 30 new firms per million people are created each year in the treated industries. This implies the shock had a relatively large impact; in New England other than Massachusetts on average 2 establishments per million people were lost each year in the treated industries after 2008. Model (6) and model (7) have point estimates that suggests there was a short term increase in entrepreneurial activity which supports proposition 4. However neither model can reject the two estimates being identical with 95\% confidence. 
\begin{center}
	\begin{table}[H]
		\centering
		\input{resources/main.tex}
		\caption{Impact of health reform on low capital industries}
	\end{table}
\end{center}

\section*{Conclusion}
Here we have provided evidence that a subsidy to entrepreneurial activity can have a measurable impact in the amount of new firms created. We use a model that describes entrepreneurship in terms of idea generation and entrepreneurial resources. We test the model's predictions against the shock created by the 2006 health reform act in Massachusetts. We find that entrepreneurs with low access to capital were most affected by the improvement in access to health care. This led to an increase in entrepreneurial activity in industry that required low levels of capital investment. The effect was magnified in its first year but persists as a long term phenomenon. 

\appendix
\section*{Appendix}
\subsection*{Appendix A}

Let $\Omega$ represent the set of all possible ideas. In each time period $t$, a potential entrepreneur $i$ makes a draw $\omega_{it} \subset \Omega$ from this set. To simplify assume each entrepreneur makes the same number of draws in each period. 

Assume that ideas can be totally ordered based on each of the factors. Let $X:\Omega\to I^n \subset \mathbb{R}^n$ be a random variable where $n$ is the number of factors that can impact the decision to start a company. For example, consider an idea $\omega$ to start a software company. One of the factors required might be a software developer with an advanced knowledge of computer science. The random variable would assign a real number that represents the knowledge required to the dimension $j$ of $\mathbb{R}^n$ that represents software development skill as a factor. Another idea $\omega'$ that requires less advanced software development would be assigned a number less than the first idea. 
$$X(\omega)^j > X(\omega')^j $$

Let $r_i \in I^n$ represent the resources of the entrepreneur. To continue the example, the software entrepreneur may have access through her social network to a certain level of software developer willing to help out but not easy access to more advanced developers. An idea is executed by the entrepreneur if all required factors can be met by the entrepreneur. 
$$r_i^j \ge X(\omega_{it})^j \; \forall j \in n$$

\begin{comment}
We call an idea \textit{unbounded} for an individual if the idea can be executed and \textit{single bounded} on dimension $k$ if
$$\alpha_i^k = X(\omega{it})^k \wedge  \alpha_i^j > X(\omega{it})^j \; \forall j \ne k \in n$$
\end{comment}

The probability of an idea being executed by an entrepreneur with resources $r_i$ in time $t$ is therefore 
$$Pr\left(\bigcap _{j=1}^n  r_i^j \ge X^j\right) = F_X(r)$$
Let $F_Y$ represent the distribution of entrepreneurs with resources $r$ over the space of ideas. Then the expected number of startups generated at time $t$ is then 
\begin{align}
\mathbb{E}_{Y}[F_X(r)]=\int_{I^n} F_X(r) dF_Y(r)
\end{align}

Consider how a change in the distribution of resources for one factor impacts the rate of startup creation along another factor's dimension. For example, increasing access to healthcare might impact the rate of firm creation for ideas requiring large amounts of capital differently than ideas require low amounts of capital. Let $\alpha$ represent the factor being changed, $\beta$ the factor we are measuring a reaction to, and $\Gamma$ be the set of all other factors. Let $\beta_0$ be a specific resource level on $\beta$ we are interested in and $[\underline{I_\beta}, \beta_0]$ the set of all resource levels that are equal to or less than $\beta_0$. We can write the total number of startups (6) that fall within that resource level as
\begin{align*}
& \int_{I^n} F_X(r) dF_Y(r)\\
= & \int_{I^n} \int_{\underline{I^n}}^{r} dF_X(s) dF_Y(r)\\
= & \int_{I^n} \int_{s}^{\overline{I^n}}  dF_Y(r) dF_X(s) \\
= &  \int_{I_{\Gamma}} \int_{I_{\alpha,\beta}} \int_{s}^{\overline{I^n}}  dF_Y(r)dF_{X_{\alpha,\beta}}(a,b|U)
 dF_{X_{\Gamma}}(U)\\
= &  \int_{I_{\Gamma}} \int_{I_{\alpha,\beta}} \int_{U}^{\overline{I_\Gamma}} \int_{a,b}^{\overline{I_{\alpha,\beta}}} dF_{Y_{\alpha,\beta}}(s,t|V)dF_{Y_\Gamma}(V) dF_{X_{\alpha,\beta}}(a,b|U) dF_{X_{\Gamma}}(U)\\
= &  \int_{I_{\Gamma}} \int_{U}^{\overline{I_\Gamma}} \int_{I_{\alpha,\beta}} \int_{a,b}^{\overline{I_{\alpha,\beta}}} dF_{Y_{\alpha,\beta}}(s,t|V) dF_{X_{\alpha,\beta}}(a,b|U) dF_{Y_\Gamma}(V) dF_{X_{\Gamma}}(U)\\
= &  \int_{I_{\Gamma}} \int_{U}^{\overline{I_\Gamma}} \int_{I_{\beta}} \int_{I_{\alpha}} \int_{b}^{\overline{I_{\beta}}} \int_{a}^{\overline{I_{\alpha}}} dF_{Y_{\alpha,\beta}}(s,t|V) dF_{X_{\alpha,\beta}}(a,b|U) dF_{Y_\Gamma}(V) dF_{X_{\Gamma}}(U)\\
\implies S(\beta_0) = & \int_{I_{\Gamma}} \int_{U}^{\overline{I_\Gamma}} \int_{\underline{I_\beta}}^{\beta_0} \int_{I_{\alpha}} \int_{b}^{\overline{I_{\beta}}} \int_{a}^{\overline{I_{\alpha}}} dF_{Y_{\alpha,\beta}}(s,t|V) dF_{X_{\alpha,\beta}}(a,b|U) dF_{Y_\Gamma}(V) dF_{X_{\Gamma}}(U)
\end{align*}
Define a function $h$ that maps the conditional distribution entrepreneurs on resource $\alpha$ and a value $\beta_0$ to the rate of new enterprise creation at $\beta_0$ conditional on $\Gamma$.
\begin{align*}
h(G_{Y_\alpha},\beta_0|\Gamma) = \int_{I_{\alpha}} \int_{\beta_0}^{\overline{I_{\beta}}} \int_{a}^{\overline{I_{\alpha}}} dG_{Y_{\alpha}}(s|t) dF_{Y_{\beta}}(t) dF_{X_{\alpha,\beta}}(a,\beta_0) = \left. \frac{\partial S (\beta_0|\Gamma)}{\partial \beta} \right|_{G_{Y_{\alpha}}=F_{Y_{\alpha}}}
\end{align*}

Finally, use first order stochastic dominance to define a partial order on the set of all distributions for $G_{Y_\alpha}$. 

\textbf{Proposition 3:} 
The function $h$ has increasing differences in $G_{Y_\alpha}$ and $-\beta_0$ if $dF_{X_{\alpha,\beta}}$ is non-increasing in $\beta$. 

\textbf{Proof:}
Let $G'_{Y_\alpha}\mathop{\ge}_{\text{FOSD}} G_{Y_\alpha}$ and $\beta_0' \ge \beta_0$. Then $h$ has increasing differences in $G_{Y_\alpha}$ and $-\beta_0$ if and only if:
\begin{align*}
& h(G'_{Y_\alpha},\beta_0|\Gamma) - h(G'_{Y_\alpha},\beta_0'|\Gamma) \ge h(G_{Y_\alpha},\beta_0|\Gamma) - h(G_{Y_\alpha},\beta_0'|\Gamma)\\
\iff & h(G'_{Y_\alpha},\beta_0|\Gamma) - h(G_{Y_\alpha},\beta_0|\Gamma) \ge h(G'_{Y_\alpha},\beta_0'|\Gamma) - h(G_{Y_\alpha},\beta_0'|\Gamma)\\
\iff & \int_{I_{\alpha}} \int_{\beta_0}^{\overline{I_{\beta}}} \int_{a}^{\overline{I_{\alpha}}} dG'_{Y_{\alpha}}(s|t) dF_{Y_{\beta}}(t) dF_{X_{\alpha,\beta}}(a,\beta_0) \\ 
& - \int_{I_{\alpha}} \int_{\beta_0}^{\overline{I_{\beta}}} \int_{a}^{\overline{I_{\alpha}}} dG_{Y_{\alpha}}(s|t) dF_{Y_{\beta}}(t) dF_{X_{\alpha,\beta}}(a,\beta_0) \\
& \ge \int_{I_{\alpha}} \int_{\beta_0'}^{\overline{I_{\beta}}} \int_{a}^{\overline{I_{\alpha}}} dG'_{Y_{\alpha}}(s|t) dF_{Y_{\beta}}(t) dF_{X_{\alpha,\beta}}(a,\beta_0') \\
& - \int_{I_{\alpha}} \int_{\beta_0'}^{\overline{I_{\beta}}} \int_{a}^{\overline{I_{\alpha}}} dG_{Y_{\alpha}}(s|t) dF_{Y_{\beta_0}}(t) dF_{X_{\alpha,\beta}}(a,\beta_0')\\
\iff & \int_{I_{\alpha}} \int_{\beta_0}^{\overline{I_{\beta}}} \int_{a}^{\overline{I_{\alpha}}} (dG'_{Y_{\alpha}}(s|t) - dG_{Y_{\alpha}}(s|t)) dF_{Y_{\beta}}(t) dF_{X_{\alpha,\beta}}(a,\beta_0) \\ 
& \ge \int_{I_{\alpha}} \int_{\beta_0'}^{\overline{I_{\beta}}} \int_{a}^{\overline{I_{\alpha}}} (dG'_{Y_{\alpha}}(s|t) - dG_{Y_{\alpha}}(s|t)) dF_{Y_{\beta}}(t) dF_{X_{\alpha,\beta}}(a,\beta_0')\\
\iff & \int_{I_{\alpha}} \int_{\beta_0}^{\overline{I_{\beta}}} (1-G'_{Y_{\alpha}}(a|t)) - (1-G_{Y_{\alpha}}(a|t)) dF_{Y_{\beta}}(t) dF_{X_{\alpha,\beta}}(a,\beta_0) \\ 
& \ge \int_{I_{\alpha}} \int_{\beta_0'}^{\overline{I_{\beta}}} (1-G'_{Y_{\alpha}}(a|t)) - (1-G_{Y_{\alpha}}(a|t)) dF_{Y_{\beta}}(t) dF_{X_{\alpha,\beta}}(a,\beta_0') \\ 
\iff & \int_{I_{\alpha}} \int_{\beta_0}^{\overline{I_{\beta}}} (G_{Y_{\alpha}}(a|t) - G'_{Y_{\alpha}}(a|t)) dF_{Y_{\beta}}(t) dF_{X_{\alpha,\beta}}(a,\beta_0) \\ 
& \ge \int_{I_{\alpha}} \int_{\beta_0'}^{\overline{I_{\beta}}} (G_{Y_{\alpha}}(a|t) - G'_{Y_{\alpha}}(a|t)) dF_{Y_{\beta}}(t) dF_{X_{\alpha,\beta}}(a,\beta_0')
\end{align*}
$G'_{Y_\alpha} \mathop{\ge}_{\text{FOSD}} G_{Y_\alpha}$ implies $G_{Y_{\alpha}}(a|t) - G'_{Y_{\alpha}}(a|t)$ is non-negative, so 
\begin{align*}
\int_{I_{\alpha}} \int_{\beta_0}^{\overline{I_{\beta}}} (G_{Y_{\alpha}}(a|t) - G'_{Y_{\alpha}}(a|t)) dF_{Y_{\beta}}(t) dF_{X_{\alpha,\beta}}(a,\beta_0)
\end{align*}
is also non-negative since events can only have non-negative measure. Since $dF_{X_{\alpha,\beta}}$ is non-increasing in $\beta$, we have: 
\begin{align*}
& \int_{I_{\alpha}} \int_{\beta_0}^{\overline{I_{\beta}}} (G_{Y_{\alpha}}(a|t) - G'_{Y_{\alpha}}(a|t)) dF_{Y_{\beta}}(t) dF_{X_{\alpha,\beta}}(a,\beta_0)\\
& \ge \int_{I_{\alpha}} \int_{\beta_0}^{\overline{I_{\beta}}} (G_{Y_{\alpha}}(a|t) - G'_{Y_{\alpha}}(a|t)) dF_{Y_{\beta}}(t) dF_{X_{\alpha,\beta}}(a,\beta_0')
\end{align*}
So the proof would be completed if
\begin{align*}
& \int_{I_{\alpha}} \int_{\beta_0}^{\overline{I_{\beta}}} (G_{Y_{\alpha}}(a|t) - G'_{Y_{\alpha}}(a|t)) dF_{Y_{\beta}}(t) dF_{X_{\alpha,\beta}}(a,\beta_0') \\
&  \ge  \int_{I_{\alpha}} \int_{\beta_0'}^{\overline{I_{\beta}}} (G_{Y_{\alpha}}(a|t) - G'_{Y_{\alpha}}(a|t)) dF_{Y_{\beta}}(t) dF_{X_{\alpha,\beta}}(a,\beta_0')\\
\iff & \int_{I_{\alpha}} \int_{\beta_0}^{\beta_0'} (G_{Y_{\alpha}}(a|t) - G'_{Y_{\alpha}}(a|t)) dF_{Y_{\beta}}(t) dF_{X_{\alpha,\beta}}(a,\beta_0') \ge 0\\
\end{align*}
Which again is true from $G'_{Y_\alpha} \mathop{\ge}_{\text{FOSD}} G_{Y_\alpha}$ and the restriction to non-negative measures. $\Box$ \\

This modeling requires shocks need to cause first order stochastically dominated shifts in access to a resource. The health care act fulfills this requirement if it only weakly increases the probability that an individual has enough access to health insurance to execute a given idea. It cannot for example change the distribution of access to capital or the distribution of ideas. We believe this is the case as long as we exclude firms in the health care sector from our analysis since the shock could have created new opportunities for health entrepreneurship.  

The non-increasing condition can be interpreted as saying that we should not see more ideas for startups as the required amount of resources increases. Following our example, in general there should not be more ideas that require high capital than ideas that require lower capital. If this condition fails to hold then we will obverse an attenuation of the modeled effect. 

Note that the required FOSD shift in $\alpha$ is allowed to be conditional on the level of resource $\beta$ possessed by the entrepreneur. In our example we can imagine entrepreneurs with access to high levels of capital to be less effected by the health care act than entrepreneurs with access to low levels of capital.  

Next, consider how an increase in the distribution of $F_{Y_\alpha}$ impacts ideas from previous periods if those ideas are retained by the entrepreneur. Each period $T$'s expected number of startups becomes
\begin{align}
\mathbb{E}^T_Y[F_{X}] = \mathbb{E}_{Y_T}[F_{X}] + \sum_{t=0}^{T-1} max(0,\mathbb{E}_{Y_T}[F_{X}] - max(\mathbb{E}_{Y_i}[F_{X}] \forall i \in T{-}1,\dots,t)))
\end{align}

\textbf{Proposition 4:} If in any three consecutive periods $t=0,1,2$
\begin{enumerate}[(a)]
\item $F_{Y_\alpha,t=1} \mathop{\ge}_{\text{FOSD}} F_{Y_\alpha,t=0} $
\item $F_{Y_\alpha,t=1} = F_{Y_\alpha,t=2} $
\item $F_{Y_-\alpha|\alpha,t=0} = F_{Y_-\alpha|\alpha,t=1} =F_{Y_-\alpha|\alpha,t=2}$
\end{enumerate}
then
\begin{enumerate}[(a)]
\setcounter{enumi}{3}
\item $\mathbb{E}_{Y}[F_{Y_\alpha,t=1}] \ge \mathbb{E}_{Y}[F_{Y_\alpha,t=2}]$
\item $\mathbb{E}_{Y}[F_{Y_\alpha,t=2}] \ge \mathbb{E}_{Y}[F_{Y_\alpha,t=0}]$
\end{enumerate}

\textbf{Proof:} (d) follows from (2) and the expectation properties of first order stochastic dominance. (e) follows also from the properties of first order stochastic dominance.

\begin{comment}

\subsection*{Appendix B}

\begin{figure}[H]
	\centering
	\includegraphics[scale=.8]{resources/NAICS_99.eps}
	\caption{Total establishments with 1 to 4 employees in Massachusetts by NAICS code}
\end{figure}

In Figure 4 we see the industry classified as NAICS 99 as changing drastically versus the other industries in the data set. NAICS 99 generally refers to unclassified firms. 

\subsection*{Appendix C}

Calculating the kernel density over the entire range of Kickstarter goals provides unsatisfactory results due to extreme values. There is for example a project with a stated goal of \$100,000,000, far higher than any funded project on Kickstarter. Of the top ten funded projects on Kickstarter, the highest project goal was just \$2,000,000. \\

\begin{tabular}{|l | r| r|}
	\hline
	Project & Goal & Funding \\
	\hline
	Pebble Time	& \$500,000  & \$20,338,986\\
	Coolest Cooler & \$50,000  & \$13,285,226\\
	Pebble & \$100,000 & \$10,266,845\\
	Exploding Kittens & \$10,000 & \$8,782,571\\
	OUYA & \$950,000  & \$8,596,474\\
	Pono Music & \$800,000  & \$6,225,354\\
	Veronica Mars Movie Project & \$2,000,000  & \$5,702,153\\
	Reading Rainbow & \$1,000,000  & \$5,408,916\\
	Torment & \$900,000  & \$4,188,927\\
	\hline
\end{tabular} \\

We therefore omit projects with goals \$5,000,000 or greater from the density estimate; these projects are abnormally high relative to what is realistic on Kickstarter. They likely reflect aspirational goals that go beyond the true capital requirements of the inventor's idea. 

\end{comment}

\begin{thebibliography}{99}

\begin{comment}

\bibitem{abadie}
Abadie, Alberto. ``Synthetic Control Methods for Comparative Case Studies: Estimating the Effect of California’s Tobacco Control Program.'' Journal of the American Statistical Association 105 (2010): 493-505

\bibitem{acharya}
Acharya, Viral, and Krishnamurthy Subramanian, 2009, Bankruptcy codes and innovation, \emph{Review of Financial Studies} 22, 4949-4988.

\bibitem{kaufman}
Atkinson, Robert D., Scott Andes. 2010 ``The 2010 State New Economy Index: Benchmarking Economic Transformation in the States'' The Information Technology and Innovation Foundation, Kauffman Foundation


\bibitem{bhide}
Bhide, Amar. \emph{The origin and evolution of new businesses}. New York: Oxford UP, 2000.




\end{comment}

\bibitem{bailey}
Bailey, James B., ``Health Insurance and the Supply of Entrepreneurs: New Evidence from the Affordable Care Act's Dependent Coverage Mandate'' (March 7, 2013). 

\bibitem{bezos}
Bezos, Jeff. ``2013 Letter to Shareholders.'' Letter to Amazon shareholders. 10 Apr. 2014.

\bibitem{burton}
Burton,M.D., J. B. S{\o}rensen, C.M. Beckman. 2002. ``Coming from good stock: Career histories and new venture formation.'' M. Lounsbury, ed. \emph{Research in the Sociology of Organizations}, Vol. 19. Emerald Publishing, Bingley, UK, 229-262.

\bibitem{ericsonPakes}
Ericson, Richard, and Ariel Pakes. ``Markov-Perfect Industry Dynamics: A Framework for Empirical Work.'' \emph{The Review of Economic Studies} 62.1 (1995): 53.

\bibitem{fairlie}
Fairlie, Robert W. ``The personal computer and entrepreneurship.'' \emph{Management Science} 52.2 (2006): 187-203.

\bibitem{fairlieKarlanZinman}
Fairlie, Robert W., Dean Karlan, and Jonathan Zinman. ``Behind the GATE Experiment: Evidence on Effects of and Rationales for Subsidized Entrepreneurship Training.'' No. w17804. \emph{National Bureau of Economic Research}, 2012.

\bibitem{garthwaite}
Garthwaite, Craig, Tal Gross, and Matthew J. Notowidigdo. ``Public Health Insurance, Labor Supply, and Employment Lock.'' \emph{The Quarterly Journal of Economics} 129.2 (2014): 653-696.

\bibitem{gompers}
Gompers, P., J. Lerner, D. Scharfstein. 2005. Entrepreneurial spawning: Public corporations and the genesis of new ventures, 1986 to 1999. \emph{J. Finance} 60(2) 577-614.

\bibitem{gruber1997}
Grilber, Jonathan. ``The Incidence of Payroll Taxation: Evidence from Chile.'' \emph{Journal of Labor Economics} 15.3 pt 2 (1997).


\bibitem{gruber}
Gruber, Jonathan. ``Massachusetts points the way to successful health care reform.'' \emph{Journal of Policy Analysis and Management} 30.1 (2011): 184-192.
        
\bibitem{gruberMadrian}
Gruber, Jonathan, and Brigitte C. Madrian. ``Health insurance availability and the retirement decision.'' No. w4469. \emph{National Bureau of Economic Research}, 1993.
        
\bibitem{gm2002}
Gruber, Jonathan, and Brigitte C. Madrian. ``Health insurance, labor supply, and job mobility: a critical review of the literature''. No. w8817. \emph{National Bureau of Economic Research}, 2002.      
        
\bibitem{heimLurie}
Heim, Bradley T., and Ithai Z. Lurie. ``Does health reform affect self-employment? Evidence from Massachusetts.'' \emph{Small Business Economics} (2014): 1-14.

\bibitem{hopenhayn}
Hopenhayn, Hugo A. ``Entry, exit, and firm dynamics in long run equilibrium.'' \emph{Econometrica} (1992): 1127-1150.

\bibitem{evans}
Hurtado, Steven. ``Over Half of North American Software Developers are Moonlighting.'' \emph{Evans Data Corporation}. Nov 2012. Web. 27 Apr. 2015

\bibitem{jovanovic}
Jovanovic, Boyan. ``Selection and the Evolution of Industry.'' \emph{Econometrica} 50.3 (1982): 649.

\bibitem{klepper}
Klepper, Steven. ``Entry, exit, growth, and innovation over the product life cycle.'' \emph{American Economic Review} (1996): 562-583.

\bibitem{npr}
Knox, Richard. "Romney’s Mission: Massachusetts Health Care." \emph{National Public Radio} 8 (2006).

\bibitem{masshealth}
Massachusetts Medical Policy Institute. ``MassHealth: The Basics.'' Rep. Apr. 2014. 8 May 2014 


\begin{comment}

% <http://bluecrossmafoundation.org/sites/default/files/download/publication/PDF%20National%20comparisons%20chartpack%20june%202012.pdf/>. 


\bibitem{himmelstein}
Himmelstein, David U., Deborah Thorne, and Steffie Woolhandler. ``Medical Bankruptcy in Massachusetts: Has Health Reform Made a Difference?'' The American Journal of Medicine 124 (2011): 224-28.

\bibitem{nvca}
National Venture Captial Association. ``Q3 2014 Regional Data: Investments by State'' http://nvca.org/research/venture-investment/

\bibitem{sorensen}
Sorensen, J. B., and M. A. Fassiotto. ``Organizations as Fonts of Entrepreneurship.'' \emph{Organization Science} 22 (2011): 1322-331.



\end{comment}

\bibitem{miller}
Miller, Sarah. ``The impact of the Massachusetts health care reform on health care use among children.'' \emph{The American Economic Review} 102.3 (2012): 502-507.

\bibitem{heritage}
Owcharenko, Nina, and Robert Moffit. ``The Massachusetts Health Plan: Lessons for the States.'' \emph{The Heritage Foundation}. 18 June 2006. Web. 27 Apr. 2015.

\bibitem{roy}
Roy, Andrew Donald. ``Some thoughts on the distribution of earnings.'' \emph{Oxford economic papers} 3.2 (1951): 135-146.


\bibitem{scheuer}
Scheuer, Florian. ``Entrepreneurial Taxation with Endogenous Entry.'' \emph{American Economic Journal: Economic Policy} 6.2 (2014): 126-63.

\bibitem{schumpeter}
Schumpeter, Joseph A. \emph{Capitalism, socialism, and democracy}. New York: Harper, 1950. 

\bibitem{sorensenSorensen}
Sorensen, Jesper B., and Olav Sorenson. ``From conception to birth: Opportunity perception and resource mobilization in entrepreneurship.'' \emph{Advances in Strategic Management} 20 (2003): 89-117.

\bibitem{stuartSorensen}
Stuart, Toby E., and Olav Sorenson. ``Social networks and entrepreneurship.'' \emph{Handbook of entrepreneurship research}. Springer US, 2005. 233-252.

\bibitem{sullivan}
Sullivan, Andrew. ``Obama's Meep Meep On Healthcare.'' Web log post. \emph{The Dish}. 16 Apr. 2014. 8 May 2014 

\bibitem{bisweek}
Symonds, William and Gleckman, Howard.  ``The Health-Care Crisis: States Are Rushing In.'' \emph{Business Week}. Bloomberg, 27 Nov. 2005. Web. 27 Apr. 2015.



\begin{comment}

http://dish.andrewsullivan.com/2014/04/16/obamas-meep-meep-on-health-care


\bibitem{businessweek}
Symonds, William C. ``In Massachusetts, Health Care for All?'' Bloomberg Business Week. Bloomberg, 03 Apr. 2006. Web. 26 Jan. 2015.

\bibitem{wooldridgeBook}
Wooldridge, J. M.  2002.  Econometric Analysis of Cross Section and Panel Data.  Cambridge, MA: MIT Press.

\bibitem{census}
http://www.census.gov/prod/2008pubs/p60-235.pdf

Insured if covered by any kind of health insurance for all or part of the previous year

\bibitem{globe}
http://managinghealthcarecosts.blogspot.com/2011/06/romneycare-works.html
2008 drop in uninsured and spike in costs

\bibitem{chia}
Asks if they currently have health insurance. 2008, summer, other years spring


\bibitem{individual}
https://www.power2u.org/downloads/Mass-health-care-reform-2011.pdf
Individual plan enrollment grew between 2006 and 2008

\bibitem{summers}
Mandated Benefit Incidence (Summers, 1989)


\end{comment}

\bibitem{tuzemen}
Tuzemen, Didem, and Thealexa Becker. ``Does Health-Care Reform Support Self-Employment?.'' \emph{Economic Review} Q III (2014): 5-23.

\bibitem{uzzi}
Uzzi, Brian. ``Embeddedness in the making of financial capital: How social relations and networks benefit firms seeking financing.'' \emph{American Sociological Review} (1999): 481-505.

\end{thebibliography}


\end{document}

\begin{comment}



JOB LOCK

Baker et al 2001:
Income security increases retirement rate

Fairlie 2011:
Job lock during the recession

Gruber Madrain 1993:
Job lock reduced by keeping coverage policy

Garthwait et al 2014:
Loosing healthcare increased labor supply in tennesee. 

Gruber Hanratty 1995:
Canada health plan increased employment, wages

Fairlie 2006:
Computer ownership and entrepreneurship, strong link between women and computers. Example of resource model. 

Fairlie et al 2012:
Training having low effect on those with capital constraints. Opposite to what we see here. 


Unlike similar research using bankruptcy law \cite{acharya}, universal health-care may be a shock that significantly affects the decision making of many potential entrepreneurs. 

Ericson Pakes 1995:
Looks at firm success early in time, based on factors such as competition. Doesn't cover firm creation itself other than a sunk cost to be accounted for. We are looking earlier in the entrepreneur's thought process. 

Whinston 1983:
Moral hazard not to work, pool rather than custom disability benefits

Job lock lacks modeling of ideas. 
ideas models lack model of entrepeneru resources

The idea of venture opportunity suggests that a cost benefit analysis determines the decision of whether to leave the firm \cite{sorensen}. A shock that affects the choice of leaving the firm can reveal much about the motivations behind entrepreneurial activity.

Hopenhayn 1992:
Theory model about new and existing firms making entry and exit decisions based on NPV of firm entry costs and profit. Can nest our model before this. 

Jovanovic 1982: 
Deals with firms learning about their cost function after entry. Not sure how to make it more relevant. 

Klepper 1996: 
Firm heterogeneity in innovative capabilities and scale (appropiability) lead to differences in firm exit behavior. Draws on innovation are dependent on firm innovative capacity and R and D spend. Also has firms imitating other's innovation. Closer to "take product to market" than what we are interested here. 




HEALTH POLICY

HP: LABOR MARKET
HP: ENTREPRENEURSHIP
HP: MASSHEALTH



Bond White 2013:
Masshealth primarcy care usage: Use 2007 as treatment start, refer to uninsured rate dropping

Chan et al 2014:
MassHealth on VHA utilization. Mid 2007 as start. 

Lasser et al 2014:
MassHealth on readmissions, uses 2nd quarter 2008 as first post reform period

Long Dahlen 2014:
Masshealth on low income children. Use 2008 as post-reform period

Long et al 2012:
Coverage impact of masshealth, again use 2008 as comparions, less clear on dates than above. 

Miller 2012:
(ECON JOURNAL) MassHealth on health care use among children. Uses 2008 as post treatment year, 2006 to 2007 as implementation year. 

Miller 2012:
Utilization, again using 2008 as post treatment year

Santy et al 2014: 
Use 2009 as treatment year since individual mandate penalty was not fully started then. Maybe tie back to setup of connector service. 

Sommers et al 2014:
Uses 2007 as post reform. Changes in mortality. Not clear that reform has impact until uninsurance rates go down. 

Toussaint et al 2014: 
MassHealth on uninsured rate, use 2008 as post reform period

Tuzemen Becker 2014: 
Reform support self-employment, 2008 as post period


Arrieta 2013:
Check impact on unpaid medical bills. Use 2006 as treatment start. 

Badding Stephenson Yeoh 2012:
Impact of masshealth on bankruptcy. Use 2006 as treatment start. Refer to the drop in uninsurance rate but that rate really gets impacted in 2008. 

Baicker et al 2013: 
Impact of Oregon health on employment rate

Chetty Saez 2010:
What is the optimal insurance policy, moral hazard etc. 

clack et al 2014:
Mass health women's access, Use sept 2007 as start.

Courtemanche 2014:
MassHealth self-assessed health, July 2007 as a start via individual mandate

Dhingra 2013:
Masshealth on coverage rates. 2007 as treatment, using states as control. Contrast with Mass being special for entrepreneurship


\end{comment}

