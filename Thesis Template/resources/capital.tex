{
\def\sym#1{\ifmmode^{#1}\else\(^{#1}\)\fi}
\begin{tabular}{l*{4}{c}}
\hline\hline
          &\multicolumn{1}{c}{(1)}&\multicolumn{1}{c}{(2)}&\multicolumn{1}{c}{(3)}&\multicolumn{1}{c}{(4)}\\
\hline
Low Capital $\times$ Post 2007&  -41.768***&            &            &            \\
          & (11.716)   &            &            &            \\
MA $\times$ Low Capital $\times$ Post 2007&            &  -39.330***&  -28.867***&  -20.084***\\
          &            & (11.981)   &  (8.286)   &  (7.772)   \\
County $\times$ Year FE &       No   &      Yes   &      Yes   &      Yes   \\
Industry $\times$ Year FE &       No   &      Yes   &      Yes   &      Yes   \\
County $\times$ Industry FE &       No   &      Yes   &      Yes   &      Yes   \\
Year FE   &      Yes   &       No   &       No   &       No   \\
County FE &      Yes   &       No   &       No   &       No   \\
State FE  &      Yes   &       No   &       No   &       No   \\
Industry FE &      Yes   &       No   &       No   &       No   \\
\hline
\(N\)     &     8676   &    26040   &    11124   &    16824   \\
\(R^{2}\) &    0.023   &    0.391   &    0.601   &    0.649   \\
Control   &Within MA   &New England   &Border Counties   &    Synth   \\
\hline\hline
\multicolumn{5}{l}{\footnotesize Standard errors in parentheses}\\
\multicolumn{5}{l}{\footnotesize * p<0.10, ** p<0.05, *** p<0.01}\\
\multicolumn{5}{l}{\footnotesize Column (1) is a diff-in-diff model of non-employers per million people in Massachusetts from  }\\ 
\multicolumn{5}{l}{\footnotesize \space 2000 to 2012 with treatment after 2007 and treatment intensity equal to percentage of non-profit}\\ 
\multicolumn{5}{l}{\footnotesize \space firms within industry.}\\ 
\multicolumn{5}{l}{\footnotesize Low capital industries defined by 4 digit NAICS industries with high percentage of firms }\\
\multicolumn{5}{l}{\footnotesize \space requiring less \$1000 of startup capital according to Survey of Business Owners 2007 Public Use }\\
\multicolumn{5}{l}{\footnotesize \space Microdata Sample }\\
\multicolumn{5}{l}{\footnotesize Maine, Connecticut, Vermont, Rhode Island and New Hampshire used as controls in column (2) }\\
\multicolumn{5}{l}{\footnotesize \space under triple diff model. }\\
\multicolumn{5}{l}{\footnotesize Column (3) restricts the dataset to counties that border other states across Massachusetts, }\\
\multicolumn{5}{l}{\footnotesize \space Connecticut, Vermont, Rhode Island, New Hampshire and New York. }\\
\multicolumn{5}{l}{\footnotesize Column (4) uses a synthetic control model that matches each Massachuetts industry by county }\\
\multicolumn{5}{l}{\footnotesize \space pre-trend against US counties with similar income, age, urban and insurance rate characteristics. }\\
\end{tabular}
}
