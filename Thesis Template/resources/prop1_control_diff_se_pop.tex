{
\def\sym#1{\ifmmode^{#1}\else\(^{#1}\)\fi}
\begin{tabular}{l*{3}{c}}
\hline\hline
          &\multicolumn{1}{c}{(1)}&\multicolumn{1}{c}{(2)}&\multicolumn{1}{c}{(3)}\\
\hline
Massachusetts $\times$ Post 2007&    0.075** &    0.063   &    0.040   \\
          &  (0.030)   &  (0.037)   &  (0.031)   \\
[1em]
Year FE   &      Yes   &      Yes   &      Yes   \\
[1em]
State FE  &      Yes   &      Yes   &      Yes   \\
[1em]
County FE &      Yes   &      Yes   &      Yes   \\
\hline
\(N\)     &      804   &      264   &      336   \\
\(R^{2}\) &    0.050   &    0.136   &    0.109   \\
Control   &New England   &Border Counties   &Synthetic   \\
Weight    &Population   &Population   &Population   \\
\hline\hline
\multicolumn{4}{l}{\footnotesize Standard errors in parentheses}\\
\multicolumn{4}{l}{\footnotesize * p<0.10, ** p<0.05, *** p<0.01}\\
\multicolumn{4}{l}{\footnotesize Diff-in-diff model of health care reform from 2000 to 2012 with Massachusetts treated after 2007. }\\
\multicolumn{4}{l}{\footnotesize Maine, Connecticut, Vermont, Rhode Island and New Hampshire used as controls in column (1). }\\
\multicolumn{4}{l}{\footnotesize Column (3) uses a synthetic control model that matches each Massachuetts county pre-trend against}\\
\multicolumn{4}{l}{\footnotesize \space US counties with similar income, age, urban and insurance rate characteristics. }\\
\end{tabular}
}
