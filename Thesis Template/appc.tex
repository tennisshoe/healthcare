\chapter{Monte-Carlo Simulation}

In order to construct confidence intervals given the issues with our small group size, we first use a Monte-Carlo simulation to test various approaches. Our data generating process is based on a modified version of model (1) allowing for auto-correlated residuals and state level shocks. 
\begin{comment}
\begin{align}
Change_{sct} & = \alpha_c + \gamma_s + \delta_t + \beta \, \mathbf{1}\{\text{county in MA}\} \cdot \mathbf{1}\{\text{year > 2007}\} + \xi_{sct} \\
\xi_{sct} & = 
\begin{cases}\rho_{sc} \, \xi_{sc(t-1)} + (1-\rho_{sc}) \epsilon_{sct} &\mbox{if } t > 1 \nonumber \\ 
\epsilon_{sct}  & \mbox{if } t = 1. \end{cases}\\
\epsilon_{sct} & = \mu_{st} + \upsilon_{sct}
\end{align}
Here $\rho$ is a random variable with distribution $\mathcal{F}$, $\mu$ is a random variable with distribution $\mathcal{G}$ and $\upsilon$ is a random variable with distribution $\mathcal{H}$. We estimate $\hat{\rho}_{sc}$ for each county in New England by regressing $Change_{sct}$ against $Change_{sc(t-1)}$ in our self-employed dataset. $\mathcal{F}$ is then approximated by taking uniform draws from the set of $\hat{\rho}_{sc}$'s. We group the residuals from the $\hat{\rho}_{sc}$ estimation by state and divide them by $1/(1-\hat{\rho}_{sc})$ to generate a set of $\hat{\epsilon}_{sct}$ values. We use the mean of $\hat{\epsilon}_{sct}$ values grouped by state and year to produce a set of $\hat{\mu}_{st}$ values that once standardized appropriate $\mathcal{G}$. Then $\hat{\epsilon}_{sct}$ minus the corresponding $\hat{\mu}_{st}$ values produce a set of $\hat{\upsilon}_{sct}$ draws that once standardized approximate $\mathcal{H}$.
\end{comment}
\begin{align}
\label{eq:mc_dist}
Change_{sct} & = \alpha_c + \gamma_s + \delta_t + \beta \, \mathbf{1}\{\text{county in MA}\} \cdot \mathbf{1}\{\text{year > 2007}\} + \xi_{sct} \\
\xi_{sct} & = 
\begin{cases}\rho_{sc} \, \xi_{sc(t-1)} + (1-\rho_{sc}) \epsilon_{sct} &\mbox{if } t > 1 \nonumber \\ 
\epsilon_{sct}  & \mbox{if } t = 1. \end{cases}
\end{align}
Here $\rho$ is a random variable with distribution $\mathcal{F}$ and $\epsilon$ is a random variable with distribution $\mathcal{G}_s$ that varies by state. We estimate $\hat{\rho}_{sc}$ for each county in New England by regressing $Change_{sct}$ against $Change_{sc(t-1)}$ in our self-employed dataset. $\mathcal{F}$ is then approximated by taking uniform draws from the set of $\hat{\rho}_{sc}$'s. We group the residuals from the $\hat{\rho}_{sc}$ estimation by state and divide them by $1/(1-\hat{\rho}_{sc})$ to similarly create approximations for the $\mathcal{G}_s$ set. 

In order to test for Type 1 error, we set $\beta$ to zero in model (\ref{eq:mc_dist}), simulate the data and drop our first year's observations since they lack an auto-correlated term. To test Type 2 error, we set $\beta$ equal to a third of the standard deviation of $Change_{sct}$. Through experiment we determined this value produced sufficient variation between our inference methods for comparison. We run fixed effects regressions under homeostatic, robust, stated clustered and auto-regressive error assumptions. We also calculate Newey-West errors with a 3 year and we use a 2 step block bootstrap method with re-sampling of counties within states using the scaling correction suggested by Rao and Wu \cite{rao}. 

Table \ref{tab:mc} presents the results. We see that fixed effects under AR(1) assumptions does the best overall. However the assumption of an AR(1) process was baked into our data generating process through our functional form; it may not accurately reflect our observed data. We therefore select Huber-White inference as preferred for our dataset conditional on this data generating model since robust errors do not show excessive Type 1 error and it performs slightly better than our bootstrap procedure on Type 2 error.

\begin{table}[h]
	\centering
	\caption{Monte Carlo Simulation of Type 1 and 2 Errors}
	\begin{tabular}{lrrrr} \hline \hline
		Specification &(1) & (2) & (3) & (4)  \\  
		\hline \textbf{Simulation Parameters} & & & & \\
		States &   5 &    2 & 5 & 2\\
		Counties per State &        10 &        100 & 10 & 100\\
		Years of Observations &        12 &        12 & 12 & 12 \\
		Treatment Value &        0 &        0 & $\sigma_{Y}/3$ & $\sigma_{Y}/3$\\
		Draws &       200 & 200 & 200 & 200\\
		\hline \multicolumn{5}{l}{\textbf{Percent Rejecting Null Hypothesis}} \\
		Fixed Effects OLS &      20\% &      40\%& 49\%& \textbf{96}\%\\
		FE Huber-White &       \textbf{2\%} &       \textbf{2\%}& 44\%& 67\%\\
		FE Clustered by State &         29\% &         50\%& 62\% & 51\%\\
		FE AR(1) Process &         \textbf{2\%} &         \textbf{0\%}& 28\%& \textbf{98\%}\\
		Panel Newey-West &         9\% &         18\%& 59\% & 84\%\\
		Block Bootstrap &        \textbf{2\%} &         \textbf{2\%}& 33\%& 67\%\\
		\hline \hline 
		\multicolumn{5}{l}{\footnotesize 95\% confidence interval used to test null hypothesis. Bold  }\\
		\multicolumn{5}{l}{\footnotesize indicates values that either do not over-reject null when true }\\
		\multicolumn{5}{l}{\footnotesize  treatment is zero or do not under-reject null when true  }\\
		\multicolumn{5}{l}{\footnotesize treatment is non-zero. }\\
	\end{tabular}	
	\label{tab:mc}
\end{table}

Next we consider a slightly different model with state level errors. 
\begin{align}
\label{eq:mc_state}
	Change_{sct} & = \alpha_c + \gamma_s + \delta_t + \beta \, \mathbf{1}\{\text{county in MA}\} \cdot \mathbf{1}\{\text{year > 2007}\} + \xi_{sct} \\
	\xi_{sct} & = 
	\begin{cases}\rho_{sc} \, \xi_{sc(t-1)} + (1-\rho_{sc}) \epsilon_{sct} &\mbox{if } t > 1 \nonumber \\ 
		\epsilon_{sct}  & \mbox{if } t = 1. \end{cases} \nonumber \\
 	\epsilon_{sct} & = \upsilon_{st} + \mu_{ct}		\nonumber
\end{align}
$\rho$ is a random variable with distribution $\mathcal{F}$, $\upsilon$ has distribution $\mathcal{H}$ and $\mu$ has distribution $\mathcal{I}$. We again estimate $\hat{\rho}_{sc}$ and $\hat{\epsilon}_{sct}$ as for model (\ref{eq:mc_dist}), with $\mathcal{F}$ appropriated by the empirical distribution of $\hat{\rho}_{sc}$. Then we use the fact that for a given state $s'$ and time $t'$
$$\mathbb{E}[\epsilon_{sct}|s=s', t=s'] = \mathbb{E}[\upsilon_{st}|s=s', t=s'] + \mathbb{E}[\mu_{ct}|s=s', t=s']=v_{s't'}$$
to recover estimates $\hat{\upsilon}_{st}$ by using the set of county observations for each state in each time period. Then we produce estimates  $\hat{\mu}_{ct}$ by subtracting $\hat{\upsilon}_{st}$ from $\hat{\epsilon}_{sct}$ for each observation. The empirical distribution of $\hat{\upsilon}_{st}$ and $\hat{\mu}_{ct}$ approximate  $\mathcal{H}$ and $\mathcal{I}$. 

Results for the model (\ref{eq:mc_state}) data generating process are given in Table \ref{tab:mc_state}. This model specification is a challenge for all our inference methods. We again select Huber-White inference as preferred since it performs well relative to the alternatives. 

\begin{table}[h]
	\centering
	\caption{Monte Carlo Simulation of Type 1 and 2 Errors}
	\begin{tabular}{lrrrr} \hline \hline
		Specification &(1) & (2) & (3) & (4)  \\  
		\hline \textbf{Simulation Parameters} & & & & \\
		States &   5 &    2 & 5 & 2\\
		Counties per State &        10 &        100 & 10 & 100\\
		Years of Observations &        12 &        12 & 12 & 12 \\
		Treatment Value &        0 &        0 & $\sigma_{Y}/20$ & $\sigma_{Y}/20$\\
		Draws &       200 & 200 & 200 & 200\\
		\hline \multicolumn{5}{l}{\textbf{Percent Rejecting Null Hypothesis}} \\
		Fixed Effects OLS &      22\% &      47\%& 53\%& 81\%\\
		FE Huber-White &       11\% &       10.5\%& 42\%& 69\%\\
		FE Clustered by State &         30\% &         49\%& 52.5\% & 45.5\%\\
		FE AR(1) Process &         \textbf{4.5\%} &         23\%& 23.5\%& 90.5\%\\
		Panel Newey-West &         21.5\% &         29.5\%& 53.5\% & 77\%\\
		Block Bootstrap &        7\% &         12.5\%& 32.5\%& 69\%\\
		\hline \hline 
		\multicolumn{5}{l}{\footnotesize 95\% confidence interval used to test null hypothesis. Bold  }\\
		\multicolumn{5}{l}{\footnotesize indicates values that either do not over-reject null when true }\\
		\multicolumn{5}{l}{\footnotesize  treatment is zero or do not under-reject null when true  }\\
		\multicolumn{5}{l}{\footnotesize treatment is non-zero. }\\
	\end{tabular}	
	\label{tab:mc_state}
\end{table}

\clearpage
\newpage
